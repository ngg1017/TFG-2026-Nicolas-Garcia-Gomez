\apendice{Plan de Proyecto}

\section{Introducción}

En este apartado se describe de forma breve el propósito general del proyecto, su contexto y el alcance del trabajo realizado, indicando qué problema se aborda y qué se pretende conseguir con el TFG.


\section{Metodología de ingeniería del software}
Si procede.
Se debe describir la metodología seguida durante el desarrollo del proyecto. En caso de existir desarrollo software, se indicará la metodología utilizada (por ejemplo, cascada, ágil o experimental)
En cada apartado de los apéndices (A hasta el G) se deberá indicar que tarea o fase de la metodología utilizada se está realizando, o bien en este nuevo apéndice deberá quedar reflejado con detalle las tareas y fases que se han ido realizando.
Si el TFG no contiene  todo el ciclo de vida de la metodología utilizada, deberá indicarse las fases y tareas que han quedado pendientes.
Si el TFG no está centrado el desarrollo software, pero puede servir para un posterior desarrollo software, se deberá indicar qué metodología de desarrollo se recomienda y argumentar los motivos.


\section{Planificación temporal}

Se presentará la planificación temporal del proyecto, preferiblemente mediante un cronograma o diagrama de Gantt, donde se reflejen las principales tareas, fases y su duración aproximada.

\subsection{Planificación económica}

\subsection{Viabilidad legal}

Si corresponde, incluir aquí el documento de aprovación por parte de la Comisión de Bioética.

Ojo \footnote{Los anexos deben de tener su propia bibliografía, eso es tan fácil como utilizar referencias igual que en la memoria \cite{bortolot2005}}

