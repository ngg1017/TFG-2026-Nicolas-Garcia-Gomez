\capitulo{4}{Metodología}

\section{Descripción de los datos.}
Se describirán los datos utilizados en el proyecto, indicando su origen, tipo, formato y características principales. 
Cuando el tratamiento de datos sea extenso, se hará referencia al anexo correspondiente.

\section{Técnicas y herramientas.}

Se presentarán las técnicas metodológicas y las herramientas empleadas durante el desarrollo del proyecto. 
Se deben justificar las elecciones realizadas frente a otras posibles alternativas.

Esta parte de la memoria tiene como objetivo presentar las técnicas metodológicas y las herramientas de desarrollo que se han utilizado para llevar a cabo el proyecto. Si se han estudiado diferentes alternativas de metodologías, herramientas, bibliotecas se puede hacer un resumen de los aspectos más destacados de cada alternativa, incluyendo comparativas entre las distintas opciones y una justificación de las elecciones realizadas. 
No se pretende que este apartado se convierta en un capítulo de un libro dedicado a cada una de las alternativas, sino comentar los aspectos más destacados de cada opción, con un repaso somero a los fundamentos esenciales y referencias bibliográficas para que el lector pueda ampliar su conocimiento sobre el tema.


\section{Metodología de Ingeniería del Software}

En caso de que el TFG incluya desarrollo software, se indicará la metodología de ingeniería del software seguida y se describirá cómo se han organizado las fases y tareas del proyecto.

Resumir la metodología adoptada y la referencia al anexo correspondiente.





