\apendice{Registro de interacciones con sistemas de IA}

\section{Registro de interacciones con sistemas de IA}

Este anexo tiene como objetivo garantizar la transparencia en el uso de
herramientas de inteligencia artificial generativa durante la realización
del Trabajo de Fin de Grado. En él se deberá incluir un registro
ordenado de los \textit{prompts}, solicitudes, o peticiones formuladas por el estudiante en las interacciones con sistemas de IA que hayan sido más significativas en relación al desarrollo del TFG. Asimismo se incluirán bien las respuestas generadas directamente por los sistemas de IA, o bien se reflejará la importancia que ha tenido esa interacción en el desarrollo del trabajo ( código, texto, enfoque...) de esas interacciones significativas.

El propósito de este anexo no es penalizar el uso de inteligencia artificial,
sino documentarlo de forma clara, permitiendo al tribunal evaluar el alcance,
la finalidad y el impacto real de dichas herramientas en el proceso de
aprendizaje y en los resultados obtenidos. El uso de IA debe entenderse como
una herramienta de apoyo, nunca como un sustituto del razonamiento,
la toma de decisiones técnicas o la autoría del trabajo.

Para cada interacción relevante con sistemas de IA, se deberá indicar, al
menos, la siguiente información:
\begin{itemize}
    \item Herramienta o modelo de IA utilizado.
    \item Fecha aproximada de uso.
    \item Prompt/petición/solicitud/forma de uso realizada por el estudiante.
    \item Respuesta completa generada por la IA (opcional)
    \item Uso posterior de dicha respuesta en el TFG (orientación conceptual,
    mejora de redacción, generación de ideas, apoyo técnico, generación de código, etc.).
\end{itemize}

Este anexo podrá estructurarse en forma de tablas o listados numerados, y
solo deberá incluir aquellas interacciones que hayan tenido una influencia
significativa en el desarrollo del trabajo. 


